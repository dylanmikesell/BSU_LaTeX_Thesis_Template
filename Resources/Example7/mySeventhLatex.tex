\documentclass{article}
\usepackage{xspace}
\newcommand{\latex}{\LaTeX\xspace}

\usepackage{graphicx} % need to include the 'graphicx' package for figures.
\graphicspath{{./Figs/}} % Tell LaTeX where it should look for your figures.

% For groups of figure folders you can do this...
% \graphicspath{{subdir1/}{subdir2/}{subdir3/}...{subdirn/}} 

\usepackage{verbatim} % this allows us to type latex commands, but not interpet them

\author{Dylan Mikesell} % define the author
\title{\latex Tutorial} % define the title
\date{01 October 2015} % set the date (for today you can use the command \today instead of explicitly writing the date.)

% end the PREAMBLE
%------------------------------------------------------------------------------
\begin{document}

\maketitle % generate the title

%------------------------------------------------------------------------------
\newpage

\tableofcontents % make a table of contents

\newpage

\section{My first section} % make a section for the table of contents

%------------------------------------------------------------------------------
% Make a figure example
%------------------------------------------------------------------------------

This is my first \latex document. And this is my first figure reference (Fig.~\ref{fig:firstFigure}).% type some words

\begin{figure}
	\centering
	\includegraphics[width=0.5\columnwidth]{LaTeXLogo.png}
	% \includegraphics[width=0.5\columnwidth]{./Figs/LaTeXLogo.png}
	\caption{This is my first figure.}
	\label{fig:firstFigure}
\end{figure}

%------------------------------------------------------------------------------
% Make a table example
%------------------------------------------------------------------------------
This is my first table (Tab.~\ref{tab:firstTable}).

% what does the \noindent command do? Uncomment line below.
% \noindent This is my first table (Tab.~\ref{tab:firstTable}).

\begin{table}[b]
	% \centering
	\begin{tabular}{l|c|r}
	$a \approx b$ & $a \ne b$ & $a = \int_{t_0}^{t_1} \frac{g(t)}{t} dt $ \\ \hline
	\textbf{column 1} & \textbf{column 2} & \textbf{column 3} \\ \hline
	\end{tabular}
	\caption{I just made my first table!}
	\label{tab:firstTable}
\end{table}

%------------------------------------------------------------------------------
% Make some equation examples
%------------------------------------------------------------------------------

We can make an inline equation with \textit{\$\$ arg \$\$}, but the equation will not have a number associated with it. e.g. $$ a = \frac{1}{N} $$

We can add an equation using the equation environmnt (e.g. eq.~\ref{eqn:equation}). 
\begin{verbatim}
	\begin{equation}
		A\mathbf{x} = \mathbf{y}
		\label{eqn:equation}
	\end{equation}
\end{verbatim}

\begin{equation}
	A\mathbf{x} = \mathbf{y}
	\label{eqn:equation}
\end{equation}

We could also make an equation using the eqnarray environment (e.g. eq.~\ref{eqn:eqnarray}).
\begin{verbatim}
	\begin{eqnarray}
		\nonumber
		A\mathbf{x} = \mathbf{y} \\
		while \ \mathbf{x}_{i} \ge 0 
		\label{eqn:eqnarray}
	\end{eqnarray}
\end{verbatim}

\begin{eqnarray}
	\nonumber
	A\mathbf{x} = \mathbf{y} \\
	while \ x_{i} \ge 0 
	\label{eqn:eqnarray}
\end{eqnarray}

%------------------------------------------------------------------------------
\newpage

\section{Adios} % make a final section

\ldots{} and that's it.

\end{document}